% Options for packages loaded elsewhere
% Options for packages loaded elsewhere
\PassOptionsToPackage{unicode}{hyperref}
\PassOptionsToPackage{hyphens}{url}
\PassOptionsToPackage{dvipsnames,svgnames,x11names}{xcolor}
%
\documentclass[
  9pt,
  custompaper,
]{article}
\usepackage{xcolor}
\usepackage[paperwidth=26in,paperheight=20in,margin=0.35in]{geometry}
\usepackage{amsmath,amssymb}
\setcounter{secnumdepth}{-\maxdimen} % remove section numbering
\usepackage{iftex}
\ifPDFTeX
  \usepackage[T1]{fontenc}
  \usepackage[utf8]{inputenc}
  \usepackage{textcomp} % provide euro and other symbols
\else % if luatex or xetex
  \usepackage{unicode-math} % this also loads fontspec
  \defaultfontfeatures{Scale=MatchLowercase}
  \defaultfontfeatures[\rmfamily]{Ligatures=TeX,Scale=1}
\fi
\usepackage{lmodern}
\ifPDFTeX\else
  % xetex/luatex font selection
\fi
% Use upquote if available, for straight quotes in verbatim environments
\IfFileExists{upquote.sty}{\usepackage{upquote}}{}
\IfFileExists{microtype.sty}{% use microtype if available
  \usepackage[]{microtype}
  \UseMicrotypeSet[protrusion]{basicmath} % disable protrusion for tt fonts
}{}
\makeatletter
\@ifundefined{KOMAClassName}{% if non-KOMA class
  \IfFileExists{parskip.sty}{%
    \usepackage{parskip}
  }{% else
    \setlength{\parindent}{0pt}
    \setlength{\parskip}{6pt plus 2pt minus 1pt}}
}{% if KOMA class
  \KOMAoptions{parskip=half}}
\makeatother
% Make \paragraph and \subparagraph free-standing
\makeatletter
\ifx\paragraph\undefined\else
  \let\oldparagraph\paragraph
  \renewcommand{\paragraph}{
    \@ifstar
      \xxxParagraphStar
      \xxxParagraphNoStar
  }
  \newcommand{\xxxParagraphStar}[1]{\oldparagraph*{#1}\mbox{}}
  \newcommand{\xxxParagraphNoStar}[1]{\oldparagraph{#1}\mbox{}}
\fi
\ifx\subparagraph\undefined\else
  \let\oldsubparagraph\subparagraph
  \renewcommand{\subparagraph}{
    \@ifstar
      \xxxSubParagraphStar
      \xxxSubParagraphNoStar
  }
  \newcommand{\xxxSubParagraphStar}[1]{\oldsubparagraph*{#1}\mbox{}}
  \newcommand{\xxxSubParagraphNoStar}[1]{\oldsubparagraph{#1}\mbox{}}
\fi
\makeatother


\usepackage{longtable,booktabs,array}
\usepackage{calc} % for calculating minipage widths
% Correct order of tables after \paragraph or \subparagraph
\usepackage{etoolbox}
\makeatletter
\patchcmd\longtable{\par}{\if@noskipsec\mbox{}\fi\par}{}{}
\makeatother
% Allow footnotes in longtable head/foot
\IfFileExists{footnotehyper.sty}{\usepackage{footnotehyper}}{\usepackage{footnote}}
\makesavenoteenv{longtable}
\usepackage{graphicx}
\makeatletter
\newsavebox\pandoc@box
\newcommand*\pandocbounded[1]{% scales image to fit in text height/width
  \sbox\pandoc@box{#1}%
  \Gscale@div\@tempa{\textheight}{\dimexpr\ht\pandoc@box+\dp\pandoc@box\relax}%
  \Gscale@div\@tempb{\linewidth}{\wd\pandoc@box}%
  \ifdim\@tempb\p@<\@tempa\p@\let\@tempa\@tempb\fi% select the smaller of both
  \ifdim\@tempa\p@<\p@\scalebox{\@tempa}{\usebox\pandoc@box}%
  \else\usebox{\pandoc@box}%
  \fi%
}
% Set default figure placement to htbp
\def\fps@figure{htbp}
\makeatother





\setlength{\emergencystretch}{3em} % prevent overfull lines

\providecommand{\tightlist}{%
  \setlength{\itemsep}{0pt}\setlength{\parskip}{0pt}}



 


\usepackage{multicol}
\usepackage{fancyhdr}
\usepackage{xcolor}
\usepackage{graphicx}
\usepackage{nopageno}
\definecolor{putiorblue}{HTML}{2563eb}
\definecolor{putiorpurple}{HTML}{7c3aed}
\definecolor{putiorgreen}{HTML}{16a34a}
\definecolor{putiororange}{HTML}{d97706}
\definecolor{lightgray}{HTML}{f3f4f6}
\pagestyle{empty}
\raggedbottom
\raggedcolumns
\setlength{\columnsep}{0.3in}
\setlength{\parskip}{0pt}
\setlength{\parindent}{0pt}
% Prevent ALL page breaks
\widowpenalty=10000
\clubpenalty=10000
\brokenpenalty=10000
\interlinepenalty=10000
\predisplaypenalty=10000
\postdisplaypenalty=10000
% Compact sections
\usepackage{titlesec}
\titlespacing*{\section}{0pt}{0.5ex}{0.3ex}
\titlespacing*{\subsection}{0pt}{0.4ex}{0.2ex}
\titleformat*{\section}{\large\bfseries}
\titleformat*{\subsection}{\normalsize\bfseries}
\makeatletter
\@ifpackageloaded{caption}{}{\usepackage{caption}}
\AtBeginDocument{%
\ifdefined\contentsname
  \renewcommand*\contentsname{Table of contents}
\else
  \newcommand\contentsname{Table of contents}
\fi
\ifdefined\listfigurename
  \renewcommand*\listfigurename{List of Figures}
\else
  \newcommand\listfigurename{List of Figures}
\fi
\ifdefined\listtablename
  \renewcommand*\listtablename{List of Tables}
\else
  \newcommand\listtablename{List of Tables}
\fi
\ifdefined\figurename
  \renewcommand*\figurename{Figure}
\else
  \newcommand\figurename{Figure}
\fi
\ifdefined\tablename
  \renewcommand*\tablename{Table}
\else
  \newcommand\tablename{Table}
\fi
}
\@ifpackageloaded{float}{}{\usepackage{float}}
\floatstyle{ruled}
\@ifundefined{c@chapter}{\newfloat{codelisting}{h}{lop}}{\newfloat{codelisting}{h}{lop}[chapter]}
\floatname{codelisting}{Listing}
\newcommand*\listoflistings{\listof{codelisting}{List of Listings}}
\makeatother
\makeatletter
\makeatother
\makeatletter
\@ifpackageloaded{caption}{}{\usepackage{caption}}
\@ifpackageloaded{subcaption}{}{\usepackage{subcaption}}
\makeatother
\usepackage{bookmark}
\IfFileExists{xurl.sty}{\usepackage{xurl}}{} % add URL line breaks if available
\urlstyle{same}
\hypersetup{
  pdftitle={putior Cheatsheet},
  colorlinks=true,
  linkcolor={blue},
  filecolor={Maroon},
  citecolor={Blue},
  urlcolor={blue},
  pdfcreator={LaTeX via pandoc}}


\title{putior Cheatsheet}
\usepackage{etoolbox}
\makeatletter
\providecommand{\subtitle}[1]{% add subtitle to \maketitle
  \apptocmd{\@title}{\par {\large #1 \par}}{}{}
}
\makeatother
\subtitle{Workflow Visualization from Code Annotations}
\author{}
\date{}
\begin{document}
\maketitle


\begin{multicols}{4}

\begin{center}
\includegraphics[width=1.2in]{../../man/figures/logo.png}

\textbf{\Large putior}

\textit{PUT + Input + Output + R}
\end{center}

\vspace{0.1cm}

Extract beautiful workflow diagrams from your code annotations. Works with R, Python, SQL, Shell, and Julia.

\section*{\textcolor{putiorblue}{Quick Start}}

\begin{verbatim}
# 1. Add annotation
#put label:"Load Data",
     output:"clean.csv"

# 2. Generate diagram
library(putior)
put_diagram(put("./"))
\end{verbatim}

\section*{\textcolor{putiorblue}{Annotation Syntax}}

\subsection*{Basic Format}
\begin{verbatim}
#put key:"value", key:"value"
\end{verbatim}

\subsection*{Minimal (label only)}
\begin{verbatim}
#put label:"My Step"
\end{verbatim}
\textit{ID auto-generated, type = "process"}

\subsection*{Full Annotation}
\begin{verbatim}
#put id:"step1", \
     label:"Load Data", \
     node_type:"input", \
     input:"config.json", \
     output:"data.csv"
\end{verbatim}

\subsection*{Alternative Formats}
\begin{verbatim}
#put label:"Step"   # Standard
# put label:"Step"  # Space
#put| label:"Step"  # Pipe
#put: label:"Step"  # Colon
\end{verbatim}

\section*{\textcolor{putiorpurple}{Node Types \& Shapes}}

\begin{tabular}{|l|l|l|}
\hline
\textbf{Type} & \textbf{Shape} & \textbf{Use For} \\
\hline
input & \texttt{([ ])} & Data sources \\
process & \texttt{[ ]} & Transforms \\
output & \texttt{[[ ]]} & Reports \\
decision & \texttt{\{ \}} & Branching \\
start & \texttt{([orange])} & Entry \\
end & \texttt{([green])} & Exit \\
\hline
\end{tabular}

\subsection*{Annotation Fields}

\begin{tabular}{|l|l|l|}
\hline
\textbf{Field} & \textbf{Req?} & \textbf{Default} \\
\hline
id & No & Auto UUID \\
label & Rec. & None \\
node\_type & No & "process" \\
input & No & None \\
output & No & File name \\
\hline
\end{tabular}

\subsection*{File Artifacts}
\begin{verbatim}
# Multiple files
output:"data.csv, log.txt"

# Variable tracking
output:"result.internal"
\end{verbatim}
\textit{.internal = in-memory only}

\subsection*{Connecting Scripts}
\begin{verbatim}
# Script A outputs file
#put label:"Fetch",
     output:"data.csv"

# Script B reads that file
#put label:"Process",
     input:"data.csv"
\end{verbatim}

\section*{\textcolor{putiorgreen}{Key Functions}}

\subsection*{Core Workflow}
\begin{verbatim}
# Extract annotations
workflow <- put("./src/")
workflow <- put("script.R")
workflow <- put("./",
                recursive = TRUE)

# Generate diagram
put_diagram(workflow)
put_diagram(workflow,
            theme = "github")
\end{verbatim}

\subsection*{Auto-Annotation}
\begin{verbatim}
# Auto-detect from code
put_auto("./src/")

# Generate annotation text
put_generate("./src/")
put_generate("./src/",
  output = "clipboard")

# Merge manual + auto
put_merge("./src/",
  merge_strategy = "supplement")
\end{verbatim}

\subsection*{Output Options}
\begin{verbatim}
# Console (default)
put_diagram(wf)

# Copy to clipboard
put_diagram(wf,
  output = "clipboard")

# Save to file
put_diagram(wf,
  output = "file",
  file = "diagram.md")
\end{verbatim}

\subsection*{Interactive Features}
\begin{verbatim}
# Show source file info
put_diagram(wf,
  show_source_info = TRUE)

# Clickable nodes (VS Code)
put_diagram(wf,
  enable_clicks = TRUE,
  click_protocol = "vscode")
\end{verbatim}

\section*{\textcolor{putiororange}{Diagram Options}}

\subsection*{Themes}
\texttt{light} | \texttt{dark} | \texttt{auto} | \texttt{github} | \texttt{minimal}

\begin{verbatim}
put_diagram(wf, theme="github")
\end{verbatim}

\subsection*{Directions}
\texttt{TD} (top-down) | \texttt{LR} (left-right)

\texttt{BT} (bottom-top) | \texttt{RL} (right-left)

\begin{verbatim}
put_diagram(wf, direction="LR")
\end{verbatim}

\subsection*{Visualization Modes}
\begin{verbatim}
# Simple (script connections)
put_diagram(wf)

# With data artifacts
put_diagram(wf,
  show_artifacts = TRUE)

# With file labels on edges
put_diagram(wf,
  show_files = TRUE)

# Workflow boundaries
put_diagram(wf,
  show_workflow_boundaries=TRUE)
\end{verbatim}

\end{multicols}

show\_workflow\_boundaries=TRUE) \textbackslash end\{verbatim\}

\textbackslash end\{multicols\}

\vspace{0.15cm}
\hrule
\vspace{0.15cm}

\begin{center}
\textbf{\Large Example Workflows}
\end{center}

\vspace{0.1cm}

\begin{multicols}{4}

\subsection*{Simple Linear Pipeline}

\begin{verbatim}
# 01_fetch.R
#put label:"Fetch Sales Data",
     node_type:"input",
     output:"sales.csv"

# 02_clean.py
#put label:"Clean Data",
     input:"sales.csv",
     output:"clean.csv"

# 03_report.R
#put label:"Generate Report",
     node_type:"output",
     input:"clean.csv"
\end{verbatim}

\begin{center}
\includegraphics[width=3.5in]{diagrams/linear.png}
\end{center}

\columnbreak

\subsection*{Branching \& Merging}

\begin{verbatim}
# 01_fetch_sales.R
#put label:"Fetch Sales",
     node_type:"input",
     output:"sales.csv"

# 02_fetch_customers.R
#put label:"Fetch Customers",
     node_type:"input",
     output:"customers.csv"

# 03_merge.R
#put label:"Merge Datasets",
     input:"sales.csv, customers.csv",
     output:"merged.csv"

# 04_analyze.py
#put label:"Analyze",
     input:"merged.csv",
     output:"stats.json"

# 05_report.R
#put label:"Report",
     node_type:"output",
     input:"stats.json"
\end{verbatim}

\begin{center}
\includegraphics[width=2.8in]{diagrams/branching.png}
\end{center}

\columnbreak

\subsection*{Modular source() Pattern}

\begin{verbatim}
# utils.R - Utility functions
#put label:"Data Utilities",
     node_type:"input"

# analysis.R - Uses utils
#put label:"Analysis Functions",
     input:"utils.R"

# main.R - Orchestrates both
#put label:"Main Pipeline",
     input:"utils.R, analysis.R",
     output:"results.csv"
\end{verbatim}

\begin{center}
\includegraphics[width=2.5in]{diagrams/modular.png}
\end{center}

\columnbreak

\subsection*{Decision/Branching Logic}

\begin{verbatim}
# start.R
#put label:"Load Config",
     node_type:"start",
     output:"config.json"

# check.R
#put label:"Validate Data?",
     node_type:"decision",
     input:"config.json"

# path_a.R
#put label:"Full Analysis",
     input:"config.json",
     output:"full.csv"

# path_b.R
#put label:"Quick Summary",
     input:"config.json",
     output:"summary.csv"

# end.R
#put label:"Complete",
     node_type:"end",
     input:"full.csv, summary.csv"
\end{verbatim}

\begin{center}
\includegraphics[width=2.5in]{diagrams/decision.png}
\end{center}

\end{multicols}

\vspace{0.1cm}
\begin{center}
\footnotesize putior v1.0.0 | \url{https://pjt222.github.io/putior/} | MIT License | Philipp Thoss
\end{center}




\end{document}
